\documentclass[a4j,9pt]{jsarticle}
\usepackage[dvipdfmx]{graphicx}
\usepackage{mymacro}
\usepackage[all]{xy}

% alias
\newcommand{\dom}[1]{\ensuremath{\mathrm{dom}(#1)}}
\newcommand{\codom}[1]{\ensuremath{\mathrm{codom}(#1)}}
\newcommand{\comp}{\ensuremath{\mathrm{\circ}}}
\newcommand{\id}[1]{\ensuremath{\mathit{id}_{#1}}}
\newcommand{\Set}{\ensuremath{\mathbf{Set}圏}}
\newcommand{\PoSet}{\ensuremath{\mathbf{PoSet}圏}}


% arrow
\newcommand{\aID}[1]{\ar@(ur,ul)_{id_{#1}}}

\renewcommand{\lstlistingname}{リスト}
\lstset{
escapechar=\%,%
columns=[l]{fullflexible}, % 文字をつめる
tabsize={8},%
basicstyle=\footnotesize,%
classoffset=1,%
frame=tRBl,framesep=5pt,%
showstringspaces=false,%
numbers=left,stepnumber=1,numberstyle=\footnotesize%
}

\usepackage{amsthm}
\newtheorem{definition}{定義}[section]
\newtheorem{examples}{例}[section]


\begin{document}
\title{休日カフェタイム \#1}
\author{csnagoya}
\maketitle
\section{定義1.1: 圏の定義}
\begin{enumerate}
\item 対象(object)のコレクション
\item 射(arrow または morphism)のコレクション
\item 各arrow $f$には\dom{f}と\codom{f}が存在する。
  $f : A \to B$と記述する。(Aが\dom{f}、Bが\codom{f})
\item arrow同士の合成(composition)が可能で、これを$g \comp f$と書く。
  ただし$f : A \to B$で$g : B \to C$のとき$g \comp f : A \to C$。
  このとき$h \comp ( g \comp f) = (h \comp g) \comp f$。
\item 各オブジェクト Aに対して、$\id{A} : A \to A$というarrowが必ず存在し、次の式をみたす。
  $\forall f : A \to B, \id{B} \comp f = f$かつ$f = f \comp \id{A}$。
\end{enumerate}

\[
\xymatrix{
  A \aID{A} \ar[r]|f & B \aID{B}
}
\]

\section{例1.2: \Set}\label{set}

\begin{tabular}{|c|l|}
\hline
object & 集合 \\\hline
arrow  & 集合の関数 \\\hline
composite & 集合論的な関数合成 \\\hline
\end{tabular}

\begin{enumerate}
\item \Set のオブジェクトは集合
\item \Set の $f : A \to B$は、AからBの関数
\item 関数には\dom{f}と\codom{f}が存在する
\item
\item
\end{enumerate}

\section{例1.3: \PoSet}

\begin{tabular}{|c|l|}
\hline
object & 集合と半順序関係の組 \\\hline
arrow  & 集合の関数 \\\hline
composite & 集合論的な関数合成 \\\hline
\end{tabular}


\subsection{その前に、半順序とは}
半順序(partial order)は次の条件を満す関係のこと。

\begin{enumerate}
\item 反射律 $x \le x$
\item 推移律 $x \le y$かつ$y \le z$ならば、$x \le z$
\item 反対称律 $x \le y$かつ$y \le x$ならば$x = y$
\end{enumerate}

不等号を一般化したやつ。

そして、次の関係を満す$f : A -> B$をorder preservingな関数と呼ぶ。
\[
x \le_A y ならば f(x) \le_B f(y)
\]

\subsection{圏かどうかの確認}
\begin{enumerate}
\item \PoSet のオブジェクトは集合とその集合の上の半順序関係の組
\item \PoSet の $f : A \to B$は、AからBのorder preservingな関数
\item 関数には\dom{f}と\codom{f}が存在する
\item 合成してもorder preservingなままであることを示す。

$f : A \to B$、$ g : B \to C$、$x,y \in A$について、
$f$はorder preservingなので$f(x) \le_B f(y)$。
$g$はorder preservingなので$g(f(x)) \le_C g(f(y))$。
よって$(f \comp g)(x)\le_C (f \comp g)(y)$。

\comp が結合則を満すことは\sref{set}で示されている。

\item $id_A(x)= x$がorder preservingであることを示す。
$x = id_A(x)$、$y = id_A(y)$なので、
$x \le_A y$ならば$id_A(x) \le_A id_A(y)$。

$id_A(x) = x$が
$\id{B} \comp f = f$かつ$f = f \comp \id{A}$を満すことは、
\sref{set}で示されている。
\end{enumerate}

\end{document}
